% Created 2019-11-24 Sun 15:41
% Intended LaTeX compiler: pdflatex
\documentclass[11pt]{article}
\usepackage[utf8]{inputenc}
\usepackage[T1]{fontenc}
\usepackage{graphicx}
\usepackage{grffile}
\usepackage{longtable}
\usepackage{wrapfig}
\usepackage{rotating}
\usepackage[normalem]{ulem}
\usepackage{amsmath}
\usepackage{textcomp}
\usepackage{amssymb}
\usepackage{capt-of}
\usepackage{hyperref}
\usepackage{minted}
\usepackage[a4paper, margin=2cm]{geometry}
\usepackage{indentfirst}
\usepackage[, brazilian]{babel}
\usepackage{float}
\usepackage{color, colortbl}
\usepackage{titling}
\setlength{\droptitle}{-1.5cm}
\hypersetup{ colorlinks = true, urlcolor = blue }
\definecolor{beige}{rgb}{0.93,0.93,0.82}
\definecolor{brown}{rgb}{0.4,0.2,0.0}
\author{Fernanda Guimarães -- 2016058166}
\date{}
\title{Trabalho Prático 3 -- Redes de Computadores}
\hypersetup{
 pdfauthor={Fernanda Guimarães -- 2016058166},
 pdftitle={Trabalho Prático 3 -- Redes de Computadores},
 pdfkeywords={},
 pdfsubject={},
 pdfcreator={Emacs 26.3 (Org mode 9.1.9)}, 
 pdflang={Brazilian}}
\begin{document}

\maketitle

\section{Framework escolhido}
\label{sec:orgf5b5ad2}
Foi utilizado \texttt{poll} para o servidor e \texttt{threads} no cliente. Escolhi o poll devido a
sua excelente \href{http://man7.org/linux/man-pages/man2/poll.2.html}{documentação}. Já as \texttt{threads} foram usadas para resolver o problema do
cliente de poder digitar e receber mensagens ao mesmo tempo, e por serem mais simples de
implementar.

Assim, foram utilizadas threads apenas em duas funções:
\begin{minted}[]{c++}
int get_message(int sock);
int send_input(int sock, const std::string & userName);
\end{minted}

\section{Identificadores de Mensagem}
\label{sec:org709e75c}
O cliente seta alguns delimitadores no início de cada mensagem para o servidor.

\begin{center}
\begin{tabular}{ll}
Unicode & Significado\\
\hline
PM & Mensagem privada\\
ENQ & Listar usuários\\
STX & Broadcast\\
ACK & Novo usuário\\
\end{tabular}
\end{center}

\section{Instrucões}
\label{sec:org2ec396c}
Primeiro, compile os arquivos com o comando \texttt{make}.

Após isso, execute o servidor com uma porta (no exemplo, 9000):
\begin{verbatim}
./servidor 9000
\end{verbatim}

Agora, basta criar clientes na mesma porta:
\begin{verbatim}
./cliente localhost 9000
\end{verbatim}

O endereço é agnóstico à natureza, aceitando tanto nome de domínios quanto ipv6 ou 4.
\subsection{Listar usuários}
\label{sec:orgaa27896}
Comando:
\begin{verbatim}
users
\end{verbatim}
A primeira coisa que um cliente deve fazer é digitar um nome de usuário. Antes disso,
é mantido como \texttt{undefined}.
\subsection{Mensagem privada}
\label{sec:org1d20abb}
Para mandar uma mensagem privada (\texttt{unicast}), bast digitar o comando:
\begin{verbatim}
uni;usuárioX;msg
\end{verbatim}
\subsection{\texttt{Broadcast}}
\label{sec:org9484c26}
Para mandar uma mensagem a todos usuários, incluindo a si mesmo, basta digitar o
comando:
\begin{verbatim}
all;msg
\end{verbatim}
\subsection{Sair}
\label{sec:orge84c84e}
Comando:
\begin{verbatim}
exit
\end{verbatim}
\section{Observações}
\label{sec:orga2c0550}
\begin{itemize}
\item Clientes infinitos, setados dinamicamente;
\item Tamanho máximo de mensagem \texttt{80};
\item Servidor usa \texttt{poll}, cliente usa \texttt{threads};
\item Agnóstico a \texttt{IPV6} e \texttt{IPV4}.
\item Valores de constantes se encontram no arquivo de configuração \texttt{lib/config.cpp}.
\end{itemize}
\end{document}
